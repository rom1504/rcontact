\documentclass{report}
\usepackage[francais]{babel}
\usepackage[T1]{fontenc}
\usepackage[utf8]{inputenc}
\usepackage{titlesec}
\usepackage{minted}
\usepackage{graphicx}

\usepackage[top=3.0cm]{geometry}
\usepackage{color}
\definecolor{grey}{rgb}{0.99,0.99,0.99}
\titleformat{\chapter}[hang]{\bf\huge}{\thechapter}{2pc}{}

\setcounter{secnumdepth}{3}
\setcounter{tocdepth}{3}
        
\begin{document}

\begin{titlepage}
\begin{center}
\end{center}

\flushright
\begin{center}
\bigskip
\bigskip
\bigskip
\bigskip
\bigskip
\bigskip
\bigskip
\bigskip
\bigskip
\bigskip
\bigskip
\bigskip
\huge{Projet de langage object avancé : Gestionnaire de contacts}\\
\end{center}
\begin{center}
\bigskip
\bigskip
\bigskip
\bigskip
\bigskip
\bigskip
\bigskip
\bigskip
\bigskip
Romain Beaumont et Thomas Lourseyre\\
\bigskip
\bigskip
\bigskip
\bigskip
\bigskip
\bigskip
\bigskip
\bigskip
\bigskip
\bigskip
\bigskip
\bigskip
\bigskip
\bigskip
\bigskip

\textbf{du} 12 Mars 2013  \textbf{au} 2 Avril 2013

\bigskip
\bigskip
\bigskip
\end{center}
\end{titlepage}
 
\tableofcontents


\chapter{Introduction}
\par
L'objectif de ce projet est de réaliser un gestionnaire de contact en utilisant Qt et la stl.

\chapter{Hierarchie des classes}
\par

\chapter{Fonctionnalités implémentés}
\par

\chapter{Interface graphique}
\par


\chapter{Conclusion}
\section{Résultats}
\par
Toutes les fonctionnalités demandées sont réalisés.
\section{Avancement personnel}
\par
Ce projet a permis d'apprendre mieux la partie modèle/vue de Qt.

\end{document}